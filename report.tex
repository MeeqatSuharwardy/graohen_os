\documentclass[12pt,a4paper]{article}

% Encoding / fonts (safe defaults)
\usepackage[utf8]{inputenc}
\usepackage[T1]{fontenc}

% Layout & utilities
\usepackage[a4paper,margin=1in]{geometry}
\usepackage{graphicx}
\usepackage{xcolor}
\usepackage{booktabs}
\usepackage{float}
\usepackage{caption}
\usepackage{titlesec}
\usepackage{fancyhdr}
\usepackage{enumitem}
\usepackage{tabularx}
\usepackage[hidelinks]{hyperref}

% Fancy header/footer (fix headheight warning)
\pagestyle{fancy}
\fancyhf{}
\fancyhead[L]{GrapheneOS Flasher - Implementation Report}
\fancyhead[R]{\today}
\fancyfoot[C]{\thepage}
\setlength{\headheight}{15pt}

% Colors
\definecolor{success}{RGB}{34, 139, 34}
\definecolor{warning}{RGB}{255, 165, 0}
\definecolor{info}{RGB}{30, 144, 255}

% Title formatting (FIXED)
\titleformat{\section}
  {\Large\bfseries\color{blue!70!black}}
  {\thesection}{0.75em}{}[\titlerule[0.5pt]]

\titleformat{\subsection}
  {\large\bfseries}
  {\thesubsection}{0.75em}{}

% Document metadata
\title{\textbf{GrapheneOS Flasher Project}\\
\large Implementation Report}
\author{}
\date{\today}

\begin{document}

\maketitle

\vspace{0.5cm}

\begin{abstract}
This report documents what has been implemented in the GrapheneOS Flasher project. The application consists of a desktop application built with Electron and a Python backend service for device communication and flashing operations.
\end{abstract}

\tableofcontents
\newpage

\section{Backend Service}

\subsection{Implementation}

A Python FastAPI service has been implemented at \texttt{backend/device-service/main.py}.

\subsection{API Endpoints}

The service exposes the following endpoints:

\begin{itemize}
    \item \texttt{GET /api/v1/tools/check} - Checks if ADB and Fastboot are installed
    \item \texttt{GET /api/v1/devices/adb} - Lists devices connected via ADB
    \item \texttt{GET /api/v1/devices/fastboot} - Lists devices in fastboot mode
    \item \texttt{GET /api/v1/devices/\{serial\}/props} - Retrieves device properties
    \item \texttt{POST /api/v1/devices/\{serial\}/reboot-bootloader} - Reboots device to bootloader
    \item \texttt{POST /api/v1/actions/\{action\}} - Executes actions (unlock\_bootloader, flash\_bundle, verify\_bundle)
    \item \texttt{GET /api/v1/health} - Health check endpoint
\end{itemize}

\subsection{Device Operations}

The service implements:

\begin{itemize}
    \item Device detection in ADB and Fastboot modes
    \item Device property retrieval using \texttt{adb shell getprop}
    \item Device compatibility checking against supported Pixel devices
    \item Reboot to bootloader functionality
    \item Bootloader unlock command execution
    \item Bundle path validation
    \item SHA256 checksum verification for image bundles
    \item Flashing process execution using official GrapheneOS flash scripts
\end{itemize}

\subsection{Supported Devices}

The service recognizes the following device codenames:

\begin{table}[H]
\centering
\begin{tabularx}{\textwidth}{|X|X|}
\hline
\textbf{Device Model} & \textbf{Codename} \\
\hline
Pixel 7 Pro & cheetah \\
\hline
Pixel 7 & panther \\
\hline
Pixel 7a & lynx \\
\hline
Pixel Fold & felix \\
\hline
Pixel Tablet & tangorpro \\
\hline
Pixel 8 & shiba \\
\hline
Pixel 8 Pro & husky \\
\hline
Pixel 8a & akita \\
\hline
\end{tabularx}
\caption{Supported Device Codenames}
\end{table}

\subsection{Command Execution}

The service executes ADB and Fastboot commands asynchronously with:
\begin{itemize}
    \item Timeout protection (default 30 seconds, configurable)
    \item Dry-run mode support
    \item Error handling and logging
    \item Subprocess management
\end{itemize}

\subsection{Logging}

Operations are logged to:
\begin{itemize}
    \item File: \texttt{\textasciitilde/.grapheneos-flasher/logs/flasher\_YYYYMMDD\_HHMMSS.log}
    \item Console output
\end{itemize}

\section{Desktop Application}

\subsection{Implementation}

An Electron desktop application has been implemented at \texttt{frontend/packages/desktop/}.

\subsection{Main Process}

The Electron main process (\texttt{electron/main.ts}) implements:
\begin{itemize}
    \item Python service lifecycle management (start/stop)
    \item Browser window creation
    \item IPC handlers for device operations
    \item Bundle directory selection dialog
    \item Bundle scanning functionality
\end{itemize}

\subsection{Preload Script}

The preload script (\texttt{electron/preload.ts}) exposes the device API to the renderer process:
\begin{itemize}
    \item \texttt{checkTools()} - Check ADB/Fastboot installation
    \item \texttt{listAdbDevices()} - List ADB devices
    \item \texttt{listFastbootDevices()} - List fastboot devices
    \item \texttt{getDeviceProps(serial)} - Get device properties
    \item \texttt{rebootToBootloader(serial)} - Reboot to bootloader
    \item \texttt{runAction(action, payload)} - Execute device actions
    \item \texttt{selectBundleDirectory()} - Open directory picker
    \item \texttt{scanBundles(basePath)} - Scan for bundles
\end{itemize}

\subsection{User Interface}

The React application (\texttt{src/App.tsx}) implements an 8-step wizard:

\begin{enumerate}
    \item Prerequisites - Check ADB/Fastboot installation
    \item Detect Device - Connect and identify device
    \item Reboot to Bootloader - Enter fastboot mode
    \item Unlock Bootloader - Unlock with confirmation
    \item Select Bundle - Choose GrapheneOS factory image
    \item Verify Bundle - Validate SHA256 checksum
    \item Flash GrapheneOS - Execute flashing process
    \item Post-Flash Validation - Verify installation
\end{enumerate}

\subsection{Features}

The application includes:
\begin{itemize}
    \item Tool status checking on startup
    \item Device detection with 2-second polling interval
    \item Device information display (serial, model, codename, Android version, build fingerprint)
    \item Bootloader unlock confirmation requiring "UNLOCK" text input
    \item Bundle directory browsing
    \item Bundle scanning and codename matching
    \item SHA256 verification
    \item Dry-run mode toggle
    \item Real-time log display
    \item Step navigation with validation
\end{itemize}

\section{UI Component Library}

\subsection{Implementation}

A shared React component library has been implemented at \texttt{frontend/packages/ui/}.

\subsection{Components}

The following components have been implemented:

\begin{itemize}
    \item \texttt{Wizard} - Step-by-step navigation component
    \item \texttt{DevicePanel} - Device information display
    \item \texttt{LogsPanel} - Real-time log viewer
    \item \texttt{StatusCard} - Status indicator component
    \item \texttt{Button} - Button component
    \item \texttt{Card} - Card container component
    \item \texttt{Input} - Text input component
    \item \texttt{Textarea} - Textarea component
    \item \texttt{Alert} - Alert/notification component
    \item \texttt{Badge} - Badge component
\end{itemize}

\subsection{Styling}

Components are styled using:
\begin{itemize}
    \item Tailwind CSS
    \item shadcn/ui component patterns
    \item Lucide React icons
\end{itemize}

\section{Safety Features}

\subsection{Dry Run Mode}

Operations can be executed in dry-run mode, which logs commands without executing them.

\subsection{Confirmation Gates}

\begin{itemize}
    \item Bootloader unlock requires typing "UNLOCK" in a confirmation field
    \item Warning messages displayed for data loss operations
\end{itemize}

\subsection{Device Validation}

\begin{itemize}
    \item Only devices with recognized codenames are considered supported
    \item Codename matching between detected device and selected bundle
    \item Device state validation (must be in correct mode for operations)
\end{itemize}

\subsection{Bundle Verification}

\begin{itemize}
    \item SHA256 checksum validation against \texttt{image.zip.sha256} file
    \item File existence checks for required bundle files
    \item Signature file detection (verification not yet implemented)
\end{itemize}

\section{Technical Stack}

\subsection{Frontend}

\begin{itemize}
    \item React 18
    \item TypeScript
    \item Electron
    \item Tailwind CSS
    \item shadcn/ui
    \item Vite
    \item Lucide React
\end{itemize}

\subsection{Backend}

\begin{itemize}
    \item Python 3.9+
    \item FastAPI
    \item Uvicorn
    \item Pydantic
\end{itemize}

\section{Project Structure}

The project is organized as a monorepo:

\begin{itemize}
    \item \texttt{frontend/packages/ui/} - Shared React component library
    \item \texttt{frontend/packages/desktop/} - Electron desktop application
    \item \texttt{frontend/packages/web/} - Web demo application
    \item \texttt{backend/device-service/} - Python FastAPI service
    \item \texttt{architecture/} - Architecture diagrams
\end{itemize}

\section{Documentation}

The following documentation files have been created:

\begin{itemize}
    \item \texttt{README.md} - Project overview and quick start guide
    \item \texttt{SETUP.md} - Setup instructions
    \item \texttt{backend/device-service/README.md} - Backend service documentation
\end{itemize}

\section{Testing}

\subsection{Backend Tests}

Unit tests have been implemented at \texttt{backend/device-service/test\_device\_service.py}.

\subsection{Test Execution}

Tests can be run with:
\begin{verbatim}
cd backend/device-service
python -m pytest test_device_service.py
\end{verbatim}

\section{Platform Support}

The application has been implemented to support:
\begin{itemize}
    \item macOS
    \item Linux
    \item Windows
\end{itemize}

Platform-specific setup instructions are documented in \texttt{README.md}.

\end{document}
